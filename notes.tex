\documentclass[a4paper,twocolumn,10pt]{article}
% UTF-8 source format
\usepackage[utf8]{inputenc}

% I often type this on OSX, so ignore it.
\DeclareUnicodeCharacter{00A0}{ }

% It's written in English
\usepackage[english]{babel}

% Use times font, I love it.
\usepackage{times}
\usepackage[T1]{fontenc}

% Empty line paragraphs
\usepackage{parskip}

% References
\usepackage{varioref}
\usepackage[hidelinks]{hyperref}

% Math environments, etc.
\usepackage{amsmath}
\usepackage{amssymb}

% \nicefrac
\usepackage{units}

% Double brackets
\usepackage{stmaryrd}

% Symbols for footnotes
\usepackage[symbol*]{footmisc}

% Mod without spaces
\newcommand{\Mod}[1]{\ (\text{mod}\ #1)}

% A simple theorem
\newcommand{\theorem}[1]{\textbf{(#1)}\label{theorem-#1}}

% Control enumerations
\usepackage{enumitem}


\title{Notes for M.A.~Armstrong's \textit{Groups and Symmetry}}
\author{Christian Stigen}
\date{March 2016}

\begin{document}
  \maketitle
  \section{Symmetries of the Tetrahedron}
  \paragraph{Symmetry group} Captures the rules of how symmetries combine for a
  given object.

  \paragraph{Order of operations} In the \textit{product}\footnote{Rotations,
  flips, multiplications, additions, etc. Same order as functional
  composition.} $xyz$, do $z$ first, then $y$ and finally $x$. If order doesn't
  matter in $G$, it's commutative (or \textbf{\textit{abelian}}). Remember to
  label geometric vertices.

  \section{Axioms}
  \paragraph{Group}  Set $G$ with \textit{multiplication} (addition,
  rotation, etc.) satisfying
  \begin{itemize}
    \item \textbf{\textit{associativity}}, i.e.~$(xy)z = x(yz)$
    \item \textbf{\textit{identity element}} $e \in G$ such that $xe=x=ex$
    \item \textbf{\textit{inverse}} $e \in G$ such that $x^{-1}x=e=xx^{-1}$
  \end{itemize}

  \paragraph{Properties common to all groups}
  \begin{itemize}
    \item The identify element of a group is unique.
    \item The inverse of each element of a group is unique.
  \end{itemize}

  \section{Numbers}
  \paragraph{Addition of $\mathbb{Z}, \mathbb{Q}, \mathbb{R}, \mathbb{C}$}
  \begin{itemize}
    \item Identity is zero
    \item $-x$ is the inverse
  \end{itemize}
  \paragraph{Multiplication}
  \begin{itemize}
  \item For $\mathbb{Q}-\{0\}$, $\mathbb{Q}^{\textbf{pos}}$,
    $\mathbb{R}-\{0\}$, $\mathbb{R}^{\textbf{pos}}$, $\{+1,-1\}$,
    $\mathbb{C}-\{0\}$, $\mathcal{C}$\footnote{Complex numbers of modulus 1.},
    $\{\pm 1, \pm i\}$: $e=1$ and $x^{-1}=\nicefrac{1}{x}$.
  \end{itemize}

  \paragraph{$\mathbb{Z}$ under addition modulus $n$} $e=0$,
  $x^{-1}=n-x$ for $x\ne0$, finite \textit{abelian} group and denoted
  $\mathbb{Z}_n$.

  \paragraph{$\mathbb{Z}$ under multiplication modulus $n$} Requires $n$ to be
  prime.

  \section{Dihedral Groups}
  When $n\geqslant3$ we can manufacture a plate whish has $n$ equal sides. These are
  the non-commutative \textit{dihedral rotational symmetry groups} $D_n$. E.g.
  $D_3 = \{e,r,r^2,s,rs,r^2s\}$. $x^mx^n=x^{m+n}$ and $(x^m)^n=x^{mn}$ provided
  we interpret $x^0=e$. For any multiplication table, each element in $G$
  appears only once in every given column or row.

  $r^n=e$, $s^2=e$, $sr=r^{n-1}s$, $r^{n-1}=r^{-1}$, etc.

  Each element is of form $r^a$, $r^as$ where $0\leqslant a\leqslant n-1$.

  For $k=a+_nb$, $r^ar^b=r^k$ and $r^a(r^bs)=r^ks$.  For $l=a+_n(n-b)$,
  $(r^as)r^b=r^ls$ and $(r^as)(r^bs)=r^l$ --- thus $r$ and $s$
  \textbf{\textit{generate}} $D_n$.

  The \textbf{\textit{order}} $|G|$ is the number of elements in the group. If
  $x^n=e$, then the \textit{element} $x$ has \textit{finite} order $n$ when $n$
  is the smallest such $n$.

  \section{Subgroups and Generators}
  A \textbf{\textit{subgroup}} of $G$ is a subset of $G$ which itself forms a
  group under the multiplication of $G$. For $H$ to be a subgroup of $G$,
  $H<G$:
  \begin{itemize}
    \item $xy \in G$ for any $x,y \in H$
    \item $e_H \in G$
    \item For any $x \in H$, $x^{-1} \in G$
    \item Associativity in $G$ implies the same for $H$.
  \end{itemize}

  \paragraph{Subgroup generated by $x$, or $\langle x \rangle$}
  For an element $x$ in $G$, the set of all $x^n$ is a subgroup of $G$
  (remember $x^0=e$). Finite order $m$ means $x^0=e, x^1, \ldots, x^{m-1}$.
  So order of $x\in G$ is precisely the order of $\langle x\rangle$. If
  $\langle x\rangle=G$, i.e., generates all of $G$, then $G$ is a
  \textbf{\textit{cyclic group}}.

  \paragraph{Subgroup generated by $X$} If $X<G$\footnote{$X$ is a subgroup of
  $G$.} and, for example, $r$,$s$,$r^2$,$sr$ (called \textit{words} of $X$).

  \paragraph{Theorems}
  \begin{itemize}
    \item A non-empty subset $H$ of a group $G$ is a subgroup of $G$ if and
      only if $xy^{-1}$ belongs to $H$ whenever $x$ and $y$ belong to $H$.
    \item The intersection of two subgroups of a group is itself a subgroup.
    \item Every subgroup of $\mathbb{Z}$ is cyclic. Every subgroup of a cyclic
      group is cyclic.
  \end{itemize}

  \section{Permutations}
  A \textit{permutation} is a bijection\footnote{A one-to-one mapping between
  the elements of two sets, meaning you can always go backwards as well.} from a
  set $X$ to itself (e.g.,~replace all $3$s with $1$s).  The collecticon of
  \textit{all} permutations of $X$ forms a group $S_x$ under composition of
  functions (who each perform one specific permutation). When $X$ consists of
  the first $n$ positive integers, we get the \textbf{\textit{symmetric group}}
  $S_n$ of degree $n$ and order $n!$. $S_3$ is not abelian

  $(a_1a_2\ldots a_k)$ is called a \textbf{\textit{cyclic permutation}},
  sending $a_1$ to $a_2$, $\ldots$ , $a_k$ to $a_1$. Its length is $k$ and a
  cyclic permutation of length $k$ is called a \textbf{\textit{k-cycle}}. A
  2-cycle is called a \textbf{\textit{transposition}}. Every element of $S_n$
  can be written as many such \textbf{\textit{disjoint}}, meaning no integer is
  moved by more than one of them. Therefore they are \textit{commutative}.

  \paragraph{A few tricks}
  \begin{itemize}
    \item Each \textit{element} of $S_n$ can be written as a product of cyclic
      permutations, and any cyclic permutation can be written as a product of
      transpositions: $(a_1a_2 \ldots a_k) = (a_1a_k) \ldots (a_1a_3)(a_1)(a_2)$.
      Therefore, each \textit{element} of $S_n$ can be written as a product of
      transpositions.
    \item $(ab) = (1a)(1b)(1a)$
    \item $(1k) = (k-1,k)$$\ldots$$(34) (23) (12) (23) (34)$$\ldots$$(k-1,k)$
  \end{itemize}

  \paragraph{Theorems}
  \begin{itemize}
    \item The transpositions in $S_n$ together generate $S_n$.
    \item The transpositions $(12), (13), \ldots, ({1n)}$ together generate $S_n$.
    \item The transpositions $(12), (23), \ldots, (n-1,n)$ together generate $S_n$.
    \item The transposition $(12)$ and the $n$-cycle $(12 \ldots n)$ together
      generate $S_n$.
  \end{itemize}

  Any \textit{element} $\alpha$ of $S_n$ can be written as a product of
  \textit{transpositions} in many different ways. But the number of
  transpositions is always even or always odd. If $\alpha$ \textit{can} be
  written as the product of an even number of transpositions, then its sign
  must be $+1$; for odd, it is $-1$. Therefore, by the first trick above, a
  \textit{cyclic permutation} is even precisely when its length is odd.

  \paragraph{Theorems}
  \begin{itemize}
    \item The even permutations in $S_n$ form a subgroup of order $n!/2$ called
      the \textbf{\textit{alternating group $A_n$}} of degree $n$.
    \item For $n\geqslant 3$ the 3-cycles generate $A_n$.
  \end{itemize}

  \section{Isomorphisms}
  If two multiplication tables have corresponding elements and products, they
  are \textbf{\textit{isomorphic}}.

  \paragraph{Theorem}
  Two groups $G$ and $G'$ are \textbf{\textit{isomorphic}} if there is a
  bijection $\phi$ from $G$ to $G'$ which satisfies $\phi(xy) =
  \phi{(x)}\phi{(y)}$ for all $x,y \in G$. The function $\phi$ is called an
  \textbf{\textit{isomorphism}} between $G$ and $G'$.
  This is written $G \cong G'$.

\end{document}
